\documentclass[aspectratio=169, dvipdfmx, 11pt]{beamer}
\usepackage{here, amsmath, latexsym, amssymb, bm, ascmac, mathtools, multicol, tcolorbox, subfig}

\usetheme{Luebeck}
\usecolortheme{orchid}
\usefonttheme{professionalfonts}
\useinnertheme{circles}
\useoutertheme{infolines}
\usepackage{atbegshi}

\usepackage{amsmath}
\usepackage{graphicx}

\usepackage{ascmac}
\usepackage{enumerate}
\usepackage{moreverb}
\usepackage{undline}

\ifnum 42146=\euc"A4A2
\AtBeginShipoutFirst{\special{pdf:tounicode EUC-UCS2}}
\else
\AtBeginShipoutFirst{\special{pdf:tounicode 90ms-RKSJ-UCS2}}
\fi
\setbeamertemplate{navigation symbols}{}
\renewcommand{\kanjifamilydefault}{\gtdefault}
\setbeamercolor{title}{fg=structure, bg=}
\setbeamercolor{frametitle}{fg=structure, bg=}
\setbeamertemplate{itemize item}{\small\raise0.5pt\hbox{$\bullet$}}
\setbeamertemplate{itemize subitem}{\tiny\raise1.5pt\hbox{$\blacktriangleright$}}
\setbeamertemplate{itemize subsubitem}{\tiny\raise1.5pt\hbox{$\bigstar$}}
\newcommand{\red}[1]{\textcolor{red}{#1}}
\newcommand{\green}[1]{\textcolor{green!40!black}{#1}}
\newcommand{\blue}[1]{\textcolor{blue!80!black}{#1}}

\title{条件付き確率と乗法定理}
\subtitle{高校数学A『確率』}

\begin{document}
\maketitle


\begin{frame}{目次}
    \tableofcontents
\end{frame}


\section{条件付き確率}
\begin{frame}{目次}
    \tableofcontents[currentsection]
\end{frame}


\begin{frame}{条件付き確率}
    \begin{block}{条件付き確率の定義}
	全事象を$U$とする。$2$つの事象$A,B$について,条件付き確率$P_A(B)$は
	\begin{equation}
		P_A(B)=\frac{P(A\cap B)}{P(A)}
	\end{equation}
	と表される。(ただし,$P(A)\ne 0$)
    \end{block}
\end{frame}


\begin{frame}{条件付き確率}
    \begin{block}{条件付き確率の定義}
	全事象を$U$とする。$2$つの事象$A,B$について,条件付き確率$P_A(B)$は
	\begin{equation}
		P_A(B)=\frac{P(A\cap B)}{P(A)}
	\end{equation}
	と表される。(ただし,$P(A)\ne 0$)
    \end{block}

\begin{figure}[H]
  \begin{center}
    \includegraphics[width=45mm]{kihon.png}
    \caption{ベン図(事象の関係)}
    \label{fig:graph5}
  \end{center}
\end{figure}

\end{frame}


\begin{frame}{条件付き確率}
    \begin{block}{条件付き確率の定義}
	全事象を$U$とする。$2$つの事象$A,B$について,条件付き確率$P_A(B)$は
	\begin{equation}
		P_A(B)=\frac{P(A\cap B)}{P(A)}
	\end{equation}
	と表される。(ただし,$P(A)\ne 0$)
    \end{block}

\begin{minipage}{.4\textwidth}
	\centering\includegraphics[width=5cm]{kihon.png}
\end{minipage}
	\hfill
\begin{minipage}{.5\textwidth}
	\begin{tcolorbox}[colframe=blue,
    colback=blue!10!white,
    colbacktitle=blue!40!white,
    coltitle=black, fonttitle=\bfseries,
    title=memo]
$\frac{n(A\cap B)}{n(A)}$\\
$n(A\cap B)$:事象$A\cap B$の起こる場合の数\\
$n(A)$:事象$A$の起こる場合の数
	\end{tcolorbox}
\end{minipage}

\end{frame}


\begin{frame}{条件付き確率}
    \begin{block}{条件付き確率の定義}
	全事象を$U$とする。$2$つの事象$A,B$について,条件付き確率$P_A(B)$は
	\begin{equation}
		P_A(B)=\frac{P(A\cap B)}{P(A)}
	\end{equation}
	と表される。(ただし,$P(A)\ne 0$)
    \end{block}


\begin{figure}[H]
  \begin{center}
    \includegraphics[width=45mm]{zyou1.png}
    \caption{条件付き確率のベン図(事象の関係)}
    \label{fig:graph5}
  \end{center}
\end{figure}

\end{frame}


\begin{frame}{条件付き確率}
    \begin{block}{条件付き確率の定義}
	全事象を$U$とする。$2$つの事象$A,B$について,条件付き確率$P_A(B)$は
	\begin{equation}
		P_A(B)=\frac{P(A\cap B)}{P(A)}
	\end{equation}
	と表される。(ただし,$P(A)\ne 0$)
    \end{block}

\begin{figure}[H]
  \begin{center}
    \includegraphics[width=40mm]{zyou2.png}
    \caption{条件付き確率のベン図(事象の関係)}
    \label{fig:graph5}
  \end{center}
\end{figure}

\end{frame}


\begin{frame}{条件付き確率}
    \begin{block}{条件付き確率の定義}
	全事象を$U$とする。$2$つの事象$A,B$について,条件付き確率$P_A(B)$は
	\begin{equation}
		P_A(B)=\frac{P(A\cap B)}{P(A)}
	\end{equation}
	と表される。(ただし,$P(A)\ne 0$)
    \end{block}

\begin{minipage}{.4\textwidth}
	\centering\includegraphics[width=5cm]{zyou3.png}
\end{minipage}
	\hfill
\begin{minipage}{.5\textwidth}
	\begin{tcolorbox}[colframe=blue,
    colback=blue!10!white,
    colbacktitle=blue!40!white,
    coltitle=black, fonttitle=\bfseries,
    title=memo]
$P(A)=\frac{P(A)}{1}=\frac{P(A\cap U)}{P(U)}$
	\end{tcolorbox}
\end{minipage}

\end{frame}


\section{乗法定理}
\begin{frame}{目次}
    \tableofcontents[currentsection]
\end{frame}


\begin{frame}{乗法定理}
    \begin{block}{乗法定理}
	$2$つの事象$A,B$がともに起こる確率$P(A\cap B)$は
	\begin{equation}
		P(A\cap B)=P(A)\cdot P_A(B)
	\end{equation}
	と表される。(ただし,$P(A)\ne 0$)
    \end{block}
\end{frame}


\section{演習問題}
\begin{frame}{目次}
    \tableofcontents[currentsection]
\end{frame}


\begin{frame}{演習問題}
\begin{block}{問}
 ある感染症の検査について,患者のうち$98\%$が陽性反応を示し,患者でない者が陽性反応を示す確率は$3\%$であるとする。また,検査対象者のうち,患者である確率は$0.1\%$であるとする。このとき,検査を受けて陽性となった人が実際に患者である確率を求めよ。
\end{block}
\end{frame}


\begin{frame}{演習問題}
\begin{block}{問}
 ある感染症の検査について,患者のうち$98\%$が陽性反応を示し,患者でない者が陽性反応を示す確率は$3\%$であるとする。また,検査対象者のうち,患者である確率は$0.1\%$であるとする。このとき,検査を受けて陽性となった人が実際に患者である確率を求めよ。
\end{block}

\begin{alertblock}{問われていること}
\end{alertblock}

\end{frame}


\begin{frame}{演習問題}

\begin{block}{問}
 ある感染症の検査について,患者のうち$98\%$が陽性反応を示し,患者でない者が陽性反応を示す確率は$3\%$であるとする。また,検査対象者のうち,患者である確率は$0.1\%$であるとする。このとき,検査を受けて陽性となった人が実際に患者である確率を求めよ。
\end{block}

\begin{alertblock}{問われていること}
検査で陽性となった人が実際に患者である確率は?
\end{alertblock}

\end{frame}


\begin{frame}{演習問題}

\begin{block}{問}
 ある感染症の検査について,患者のうち$98\%$が陽性反応を示し,患者でない者が陽性反応を示す確率は$3\%$であるとする。また,検査対象者のうち,患者である確率は$0.1\%$であるとする。このとき,検査を受けて陽性となった人が実際に患者である確率を求めよ。
\end{block}

\begin{alertblock}{問われていること}
検査で陽性となった人が実際に患者である確率は?
\end{alertblock}

\begin{tcolorbox}[colframe=blue,
    colback=blue!10!white,
    colbacktitle=blue!40!white,
    coltitle=black, fonttitle=\bfseries,
    title=言葉の説明]
患者:感染症に感染している人
\end{tcolorbox}

\end{frame}


\begin{frame}{演習問題}

\begin{block}{問}
 ある感染症の検査について,患者のうち$98\%$が陽性反応を示し,患者でない者が陽性反応を示す確率は$3\%$であるとする。また,検査対象者のうち,患者である確率は$0.1\%$であるとする。このとき,検査を受けて陽性となった人が実際に患者である確率を求めよ。
\end{block}

\begin{alertblock}{問われていること}
検査で陽性となった人が実際に患者である確率は?
\end{alertblock}

\begin{tcolorbox}[colframe=blue,
    colback=blue!10!white,
    colbacktitle=blue!40!white,
    coltitle=black, fonttitle=\bfseries,
    title=言葉の説明]
患者:感染症に感染している人 \\
検査で陽性が出る:一定の確率で感染症に感染していると確認できる。
\end{tcolorbox}

\end{frame}


\begin{frame}{演習問題}

\begin{block}{問}
 ある感染症の検査について,患者のうち$98\%$が陽性反応を示し,患者でない者が陽性反応を示す確率は$3\%$であるとする。また,検査対象者のうち,患者である確率は$0.1\%$であるとする。このとき,検査を受けて陽性となった人が実際に患者である確率を求めよ。
\end{block}

\begin{alertblock}{「事象」の候補}
\end{alertblock}

\end{frame}


\begin{frame}{演習問題}

\begin{block}{問}
 ある感染症の検査について,患者のうち$98\%$が陽性反応を示し,患者でない者が陽性反応を示す確率は$3\%$であるとする。また,検査対象者のうち,患者である確率は$0.1\%$であるとする。このとき,検査を受けて陽性となった人が実際に患者である確率を求めよ。
\end{block}

\begin{alertblock}{「事象」の候補}
\end{alertblock}

\begin{tcolorbox}[colframe=blue,
    colback=blue!10!white,
    colbacktitle=blue!40!white,
    coltitle=black, fonttitle=\bfseries,
    title=hint\ 問題文中にある確率]
\end{tcolorbox}

\end{frame}


\begin{frame}{演習問題}

\begin{block}{問}
 ある感染症の検査について,患者のうち$98\%$が陽性反応を示し,患者でない者が陽性反応を示す確率は$3\%$であるとする。また,検査対象者のうち,患者である確率は$0.1\%$であるとする。このとき,検査を受けて陽性となった人が実際に患者である確率を求めよ。
\end{block}

\begin{alertblock}{「事象」の候補}
\end{alertblock}

\begin{tcolorbox}[colframe=blue,
    colback=blue!10!white,
    colbacktitle=blue!40!white,
    coltitle=black, fonttitle=\bfseries,
    title=hint\ 問題文中にある確率]
・患者のうち陽性反応を示す確率は$98\%$\\
・患者でない者が陽性反応を示す確率は$3\%$\\
・検査対象者のうち,患者である確率は$0.1\%$\\
・検査を受けて陽性となった人が実際に患者である確率
\end{tcolorbox}

\end{frame}


\begin{frame}{演習問題}

\begin{block}{問}
 ある感染症の検査について,患者のうち$98\%$が陽性反応を示し,患者でない者が陽性反応を示す確率は$3\%$であるとする。また,検査対象者のうち,患者である確率は$0.1\%$であるとする。このとき,検査を受けて陽性となった人が実際に患者である確率を求めよ。
\end{block}

\begin{alertblock}{「事象」の候補}
\end{alertblock}

\begin{tcolorbox}[colframe=blue,
    colback=blue!10!white,
    colbacktitle=blue!40!white,
    coltitle=black, fonttitle=\bfseries,
    title=hint\ 問題文中にある確率]
・患者のうち,陽性反応を示す確率は$98\%$\\
・患者でない者のうち,陽性反応を示す確率は$3\%$\\
・検査対象者のうち,患者である確率は$0.1\%$\\
・検査を受けて陽性となった者のうち,患者である確率
\\
ベン図を思い出す(何を事象として$U,A,B,...$とおけば求めたい確率を求められるのか)
\end{tcolorbox}

\end{frame}


\begin{frame}{演習問題}

\begin{block}{問}
 ある感染症の検査について,患者のうち$98\%$が陽性反応を示し,患者でない者が陽性反応を示す確率は$3\%$であるとする。また,検査対象者のうち,患者である確率は$0.1\%$であるとする。このとき,検査を受けて陽性となった人が実際に患者である確率を求めよ。
\end{block}

\begin{alertblock}{「事象」の候補}
・検査対象者である。
\end{alertblock}

\end{frame}


\begin{frame}{演習問題}

\begin{block}{問}
 ある感染症の検査について,患者のうち$98\%$が陽性反応を示し,患者でない者が陽性反応を示す確率は$3\%$であるとする。また,検査対象者のうち,患者である確率は$0.1\%$であるとする。このとき,検査を受けて陽性となった人が実際に患者である確率を求めよ。
\end{block}

\begin{alertblock}{「事象」の候補}
・検査対象者である。\\
・患者である。
\end{alertblock}

\end{frame}


\begin{frame}{演習問題}

\begin{block}{問}
 ある感染症の検査について,患者のうち$98\%$が陽性反応を示し,患者でない者が陽性反応を示す確率は$3\%$であるとする。また,検査対象者のうち,患者である確率は$0.1\%$であるとする。このとき,検査を受けて陽性となった人が実際に患者である確率を求めよ。
\end{block}

\begin{alertblock}{「事象」の候補}
・検査対象者である。\\
・患者である。\\
・検査で陽性反応がでる。
\end{alertblock}

\end{frame}


\begin{frame}{演習問題}

\begin{block}{問}
 ある感染症の検査について,患者のうち$98\%$が陽性反応を示し,患者でない者が陽性反応を示す確率は$3\%$であるとする。また,検査対象者のうち,患者である確率は$0.1\%$であるとする。このとき,検査を受けて陽性となった人が実際に患者である確率を求めよ。
\end{block}

\begin{alertblock}{「事象」の候補}
・検査対象者である。\\
・患者である。\\
・検査で陽性反応がでる。\\
(否定等を考えれば他にも考えること「は」出来る。)
\end{alertblock}

\end{frame}


\begin{frame}{演習問題}

\begin{alertblock}{「事象」の候補}
・検査対象者である。\\
・患者である。\\
・検査で陽性反応がでる。\\
\end{alertblock}

\end{frame}


\begin{frame}{演習問題}

\begin{alertblock}{事象}
・検査対象者である。\\
・患者である。\\
・検査で陽性反応がでる。\\
\end{alertblock}
これらに事象としての記号($U,A,B$)を与えて,ベン図にまとめると
\end{frame}


\begin{frame}{演習問題}

\begin{alertblock}{事象}
・($U$)検査対象者である。\\
・($A$)患者である。\\
・($B$)検査で陽性反応がでる。\\
\end{alertblock}
これらに事象としての記号($U,A,B$)を与えて,ベン図にまとめると

\begin{figure}[H]
  \begin{center}
    \includegraphics[width=40mm]{q1.png}
    \caption{ベン図(事象の関係)}
    \label{fig:graph5}
  \end{center}
\end{figure}

\end{frame}


\begin{frame}{演習問題}

\begin{alertblock}{事象}
・($U$)検査対象者である。\\
・($A$)患者である。\\
・($B$)検査で陽性反応がでる。\\
\end{alertblock}

\begin{alertblock}{問われていること}
検査を受けて陽性となった者のうち,患者である確率は?
\end{alertblock}

\begin{figure}[H]
  \begin{center}
    \includegraphics[width=35mm]{q1.png}
    \caption{ベン図(事象の関係)}
    \label{fig:graph5}
  \end{center}
\end{figure}

\end{frame}


\begin{frame}{演習問題}

\begin{alertblock}{事象}
・($U$)検査対象者である。\\
・($A$)患者である。\\
・($B$)検査で陽性反応がでる。\\
\end{alertblock}

\begin{alertblock}{問われていること}
検査を受けて陽性となった者のうち,患者である確率は?$=P_B(A)$
\end{alertblock}

\begin{figure}[H]
  \begin{center}
    \includegraphics[width=35mm]{q1.png}
    \caption{ベン図(事象の関係)}
    \label{fig:graph5}
  \end{center}
\end{figure}

\end{frame}


\begin{frame}{演習問題}

\begin{alertblock}{求める値}
\begin{center}
$P_B(A)$
\end{center}
\end{alertblock}

\end{frame}



\begin{frame}{演習問題}
    \begin{block}{条件付き確率の定義}
	全事象を$U$とする。$2$つの事象$A,B$について,条件付き確率$P_A(B)$は
	\begin{equation}
		P_A(B)=\frac{P(A\cap B)}{P(A)}
	\end{equation}
	と表される。(ただし,$P(A)\ne 0$)
    \end{block}

条件付き確率の定義より,\\
\begin{equation}
P_B(A)=\frac{P(B\cap A)}{P(B)}=\frac{P(A\cap B)}{P(B)}
\end{equation}

\end{frame}


\begin{frame}{演習問題}

\begin{alertblock}{事象}
・($U$)検査対象者である。\\
・($A$)患者である。\\
・($B$)検査で陽性反応がでる。\\
\end{alertblock}

・患者のうち,陽性反応を示す確率は$98\%$\\
・患者でない者のうち,陽性反応を示す確率は$3\%$\\
・検査対象者のうち,患者である確率は$0.1\%$\\
・検査を受けて陽性となった者のうち,患者である確率

\begin{alertblock}{求める値}
\begin{center}
$\frac{P(A\cap B)}{P(B)}$
\end{center}
\end{alertblock}

\end{frame}


\begin{frame}{演習問題}

\begin{alertblock}{求める値}
\begin{center}
$\frac{P(A\cap B)}{P(B)}$\\
$P(A\cap B)=??$\\
$P(B)=??$
\end{center}
\end{alertblock}

\end{frame}


\begin{frame}{演習問題}

\begin{alertblock}{求める値}
\begin{center}
$P(A\cap B)$
\end{center}
\end{alertblock}

\begin{alertblock}{事象}
・($U$)検査対象者である。\\
・($A$)患者である。\\
・($B$)検査で陽性反応がでる。\\
\end{alertblock}

・患者のうち,陽性反応を示す確率は$98\%$\\
・患者でない者のうち,陽性反応を示す確率は$3\%$\\
・検査対象者のうち,患者である確率は$0.1\%$\\
・検査を受けて陽性となった者のうち,患者である確率

\end{frame}


\begin{frame}{演習問題}

\begin{alertblock}{求める値}
\begin{center}
$P(A\cap B)$
\end{center}
\end{alertblock}

\end{frame}


\begin{frame}{演習問題}

    \begin{block}{乗法定理}
	$2$つの事象$A,B$がともに起こる確率$P(A\cap B)$は
	\begin{equation}
		P(A\cap B)=P(A)\cdot P_A(B)
	\end{equation}
	と表される。(ただし,$P(A)\ne 0$)
    \end{block}

乗法定理より
\begin{equation}
P(A\cap B)=P(A)\cdot P_A(B)
\end{equation}

\end{frame}


\begin{frame}{演習問題}

\begin{alertblock}{事象}
・($U$)検査対象者である。\\
・($A$)患者である。\\
・($B$)検査で陽性反応がでる。\\
\end{alertblock}

・患者のうち,陽性反応を示す確率は$98\%$\\
・患者でない者のうち,陽性反応を示す確率は$3\%$\\
・検査対象者のうち,患者である確率は$0.1\%$\\
・検査を受けて陽性となった者のうち,患者である確率

\begin{alertblock}{求める値}
\begin{center}
$P(A)\cdot P_A(B)$
\end{center}
\end{alertblock}

\end{frame}


\begin{frame}{演習問題}

\begin{alertblock}{事象}
・($U$)検査対象者である。\\
・($A$)患者である。\\
・($B$)検査で陽性反応がでる。\\
\end{alertblock}

・患者のうち,陽性反応を示す確率は$98\%$\\
・患者でない者のうち,陽性反応を示す確率は$3\%$\\
・検査対象者のうち,患者である確率は$0.1\%$\\
・検査を受けて陽性となった者のうち,患者である確率

\begin{alertblock}{求める値}
\begin{center}
$P(A)\cdot P_A(B)=(0.1\cdot \frac{1}{100})\cdot (98\cdot \frac{1}{100})$
\end{center}
\end{alertblock}

\end{frame}


\begin{frame}{演習問題}
($U$)検査対象者である。\\
($A$)患者である。\\
($B$)検査で陽性反応がでる。\\

・患者のうち,陽性反応を示す確率は$98\%$\\
・患者でない者のうち,陽性反応を示す確率は$3\%$\\
・検査対象者のうち,患者である確率は$0.1\%$\\
・検査を受けて陽性となった者のうち,患者である確率

\begin{alertblock}{求める値}
\begin{center}
$P(A)\cdot P_A(B)=(0.1\cdot \frac{1}{100})\cdot (98\cdot \frac{1}{100})$
\end{center}
実際,言葉で書くと\\
$P(A)$:(検査対象者のうち)患者である確率\\
$P_A(B)$:患者である者のうち,検査で陽性反応が出る確率
\end{alertblock}
\end{frame}


\begin{frame}{演習問題}

\begin{alertblock}{求める値}
\begin{center}
$\frac{P(A\cap B)}{P(B)}$\\
$P(A\cap B)=(0.1\cdot \frac{1}{100})\cdot (98\cdot \frac{1}{100})$\\
$P(B)=??$
\end{center}
\end{alertblock}

\end{frame}


\begin{frame}{演習問題}

\begin{alertblock}{求める値}
\begin{center}
$P(B)$
\end{center}
\end{alertblock}

\end{frame}


\begin{frame}{演習問題}

\begin{block}{以下の等式を認める($\overline{A}$:$A$の余事象)}
	\begin{equation}
	P(B)=P_A(B)\cdot P(A)+P_{\overline{A}}(B)\cdot P(\overline{A})
	\end{equation}
\end{block}

\end{frame}


\begin{frame}{演習問題}

\begin{block}{以下の等式を認める($\overline{A}$:$A$の余事象)}
	\begin{equation}
	P(B)=P_A(B)\cdot P(A)+P_{\overline{A}}(B)\cdot P(\overline{A})
	\end{equation}
\end{block}
\begin{figure}[H]
  \begin{center}
    \includegraphics[width=35mm]{hai1.png}
    \caption{ベン図(事象の関係)}
    \label{fig:graph5}
  \end{center}
\end{figure}

\end{frame}


\begin{frame}{演習問題}

\begin{block}{以下の等式を認める($\overline{A}$:$A$の余事象)}
	\begin{equation}
	P(B)=P_A(B)\cdot P(A)+P_{\overline{A}}(B)\cdot P(\overline{A})
	\end{equation}
\end{block}
\begin{figure}[H]
  \begin{center}
    \includegraphics[width=35mm]{hai1.png}
    \caption{ベン図(事象の関係)}
    \label{fig:graph5}
  \end{center}
\end{figure}

$A$は「患者である」という事象であった。\\
つまり,$\overline{A}$は
\end{frame}


\begin{frame}{演習問題}

\begin{block}{以下の等式を認める($\overline{A}$:$A$の余事象)}
	\begin{equation}
	P(B)=P_A(B)\cdot P(A)+P_{\overline{A}}(B)\cdot P(\overline{A})
	\end{equation}
\end{block}
\begin{figure}[H]
  \begin{center}
    \includegraphics[width=35mm]{hai1.png}
    \caption{ベン図(事象の関係)}
    \label{fig:graph5}
  \end{center}
\end{figure}

$A$は「患者である」という事象であった。\\
つまり,$\overline{A}$は「患者ではない」という事象のこと。
\end{frame}


\begin{frame}{演習問題}

\begin{alertblock}{求める値}
\begin{center}
$P_A(B)\cdot P(A)+P_{\overline{A}}(B)\cdot P(\overline{A})$
\end{center}
\end{alertblock}

\end{frame}


\begin{frame}{演習問題}

\begin{alertblock}{求める値}
\begin{center}
$P_A(B)\cdot P(A)+P_{\overline{A}}(B)\cdot P(\overline{A})$
\end{center}
\end{alertblock}

\begin{block}{確率の性質}
全事象を$U$とする。事象$A$の確率を$P(A)$とすると,事象$A$の余事象$\overline{A}$の確率$P(\overline{A})$は
\begin{equation}
P(\overline{A})=P(U)-P(A)=1-P(A)
\end{equation}
と求まる。
\end{block}

\end{frame}


\begin{frame}{演習問題}
$P(\overline{A})$を求めたい。\\
($U$)検査対象者である。\\
($A$)患者である。\\
($B$)検査で陽性反応がでる。\\

・患者のうち,陽性反応を示す確率は$98\%$\\
・患者でない者のうち,陽性反応を示す確率は$3\%$\\
・検査対象者のうち,患者である確率は$0.1\%$\\
・検査を受けて陽性となった者のうち,患者である確率

\begin{block}{確率の性質}
全事象を$U$とする。事象$A$の確率を$P(A)$とすると,事象$A$の余事象$\overline{A}$の確率$P(\overline{A})$は
\begin{equation}
P(\overline{A})=P(U)-P(A)=1-P(A)
\end{equation}
と求まる。
\end{block}
\end{frame}


\begin{frame}{演習問題}

\begin{alertblock}{求める値}
\begin{center}
$P(\overline{A})$
\end{center}
\end{alertblock}

($U$)検査対象者である。\\
($A$)患者である。\\

・検査対象者のうち,患者である確率は$0.1\%$\\

\begin{block}{確率の性質}
全事象を$U$とする。事象$A$の確率を$P(A)$とすると,事象$A$の余事象$\overline{A}$の確率$P(\overline{A})$は
\begin{equation}
P(\overline{A})=P(U)-P(A)=1-P(A)
\end{equation}
と求まる。
\end{block}

\end{frame}


\begin{frame}{演習問題}

($U$)検査対象者である。\\
($A$)患者である。\\

・検査対象者のうち,患者である確率は$0.1\%$\\

\begin{block}{確率の性質}
全事象を$U$とする。事象$A$の確率を$P(A)$とすると,事象$A$の余事象$\overline{A}$の確率$P(\overline{A})$は
\begin{equation}
P(\overline{A})=P(U)-P(A)=1-P(A)
\end{equation}
と求まる。
\end{block}
これより,
\begin{equation}
P(\overline{A})=1-P(A)
\end{equation}

\end{frame}


\begin{frame}{演習問題}

($U$)検査対象者である。\\
($A$)患者である。\\

・検査対象者のうち,患者である確率は$0.1\%$\\

\begin{block}{確率の性質}
全事象を$U$とする。事象$A$の確率を$P(A)$とすると,事象$A$の余事象$\overline{A}$の確率$P(\overline{A})$は
\begin{equation}
P(\overline{A})=P(U)-P(A)=1-P(A)
\end{equation}
と求まる。
\end{block}
これより,
\begin{equation}
P(\overline{A})=1-P(A)=1-0.1\cdot \frac{1}{100}=99.9\cdot \frac{1}{100}
\end{equation}

\end{frame}


\begin{frame}{演習問題}

\begin{alertblock}{求めた値と求める値のまとめ}
\begin{align}
P_B(A)=\frac{P(A\cap B)}{P(B)}
\end{align}
\end{alertblock}

\end{frame}


\begin{frame}{演習問題}

\begin{alertblock}{求めた値と求める値のまとめ}
\begin{align}
P_B(A)&=\frac{P(A\cap B)}{P(B)}\\
&=\frac{(0.1\cdot \frac{1}{100})\cdot (98\cdot \frac{1}{100})}{P_A(B)\cdot P(A)+P_{\overline{A}}(B)\cdot P(\overline{A})}
\end{align}
\end{alertblock}

\end{frame}


\begin{frame}{演習問題}

\begin{alertblock}{求めた値と求める値のまとめ}
\begin{align}
P_B(A)&=\frac{P(A\cap B)}{P(B)}\\
&=\frac{(0.1\cdot \frac{1}{100})\cdot (98\cdot \frac{1}{100})}{P_A(B)\cdot P(A)+P_{\overline{A}}(B)\cdot P(\overline{A})}\\
&=\frac{(0.1\cdot \frac{1}{100})\cdot (98\cdot \frac{1}{100})}{P_A(B)\cdot P(A)+P_{\overline{A}}(B)\cdot (99.9\cdot \frac{1}{100})}
\end{align}
\end{alertblock}

\end{frame}


\begin{frame}{演習問題}

\begin{alertblock}{求める値}
\begin{center}
$P_A(B), P(A), P_{\overline{A}}(B)$
\end{center}
\end{alertblock}

\end{frame}


\begin{frame}{演習問題}

\begin{alertblock}{求めた値と求める値のまとめ}
\begin{align}
P_B(A)&=\frac{(0.1\cdot \frac{1}{100})\cdot (98\cdot \frac{1}{100})}{P_A(B)\cdot P(A)+P_{\overline{A}}(B)\cdot (99.9\cdot \frac{1}{100})}
\end{align}
\end{alertblock}

($U$)検査対象者である。\\
($A$)患者である。\\
($B$)検査で陽性反応がでる。\\

・患者のうち,陽性反応を示す確率は$98\%$\\
・患者でない者のうち,陽性反応を示す確率は$3\%$\\
・検査対象者のうち,患者である確率は$0.1\%$\\
・検査を受けて陽性となった者のうち,患者である確率

\end{frame}


\begin{frame}
($U$)検査対象者である。\\
($A$)患者である。\\
($B$)検査で陽性反応がでる。\\

・患者のうち,陽性反応を示す確率は$98\%$\\
・患者でない者のうち,陽性反応を示す確率は$3\%$\\
・検査対象者のうち,患者である確率は$0.1\%$\\
・検査を受けて陽性となった者のうち,患者である確率

\begin{align}
P_B(A)&=\frac{(0.1\cdot \frac{1}{100})\cdot (98\cdot \frac{1}{100})}{P_A(B)\cdot P(A)+P_{\overline{A}}(B)\cdot (99.9\cdot \frac{1}{100})}\\
&=\frac{(0.1\cdot \frac{1}{100})\cdot (98\cdot \frac{1}{100})}{(98\cdot \frac{1}{100})\cdot P(A)+P_{\overline{A}}(B)\cdot (99.9\cdot \frac{1}{100})}
\end{align}

\end{frame}


\begin{frame}
($U$)検査対象者である。\\
($A$)患者である。\\
($B$)検査で陽性反応がでる。\\

・患者のうち,陽性反応を示す確率は$98\%$\\
・患者でない者のうち,陽性反応を示す確率は$3\%$\\
・検査対象者のうち,患者である確率は$0.1\%$\\
・検査を受けて陽性となった者のうち,患者である確率

\begin{align}
P_B(A)&=\frac{(0.1\cdot \frac{1}{100})\cdot (98\cdot \frac{1}{100})}{(98\cdot \frac{1}{100})\cdot P(A)+P_{\overline{A}}(B)\cdot (99.9\cdot \frac{1}{100})}\\
&=\frac{(0.1\cdot \frac{1}{100})\cdot (98\cdot \frac{1}{100})}{(98\cdot \frac{1}{100})\cdot(0.1\cdot \frac{1}{100})+P_{\overline{A}}(B)\cdot (99.9\cdot \frac{1}{100})}
\end{align}

\end{frame}



\begin{frame}
($U$)検査対象者である。\\
($A$)患者である。\\
($B$)検査で陽性反応がでる。\\

・患者のうち,陽性反応を示す確率は$98\%$\\
・患者でない者のうち,陽性反応を示す確率は$3\%$\\
・検査対象者のうち,患者である確率は$0.1\%$\\
・検査を受けて陽性となった者のうち,患者である確率

\begin{align}
P_B(A)&=\frac{(0.1\cdot \frac{1}{100})\cdot (98\cdot \frac{1}{100})}{(98\cdot \frac{1}{100})\cdot(0.1\cdot \frac{1}{100})+P_{\overline{A}}(B)\cdot (99.9\cdot \frac{1}{100})}\\
&=\frac{(0.1\cdot \frac{1}{100})\cdot (98\cdot \frac{1}{100})}{(98\cdot \frac{1}{100})\cdot(0.1\cdot \frac{1}{100})+(3\cdot \frac{1}{100})\cdot (99.9\cdot \frac{1}{100})}
\end{align}

\end{frame}


\begin{frame}
($U$)検査対象者である。\\
($A$)患者である。\\
($B$)検査で陽性反応がでる。\\

・患者のうち,陽性反応を示す確率は$98\%$\\
・患者でない者のうち,陽性反応を示す確率は$3\%$\\
・検査対象者のうち,患者である確率は$0.1\%$\\
・検査を受けて陽性となった者のうち,患者である確率

\begin{align}
P_B(A)&=\frac{(0.1\cdot \frac{1}{100})\cdot (98\cdot \frac{1}{100})}{(98\cdot \frac{1}{100})\cdot(0.1\cdot \frac{1}{100})+(3\cdot \frac{1}{100})\cdot (99.9\cdot \frac{1}{100})}\\
&=\frac{0.1\cdot 98}{98\cdot 0.1\cdot 3\cdot 99.9}\\
&=0.03166397
\end{align}

\end{frame}


\begin{frame}
($U$)検査対象者である。\\
($A$)患者である。\\
($B$)検査で陽性反応がでる。\\

・患者のうち,陽性反応を示す確率は$98\%$\\
・患者でない者のうち,陽性反応を示す確率は$3\%$\\
・検査対象者のうち,患者である確率は$0.1\%$\\
・検査を受けて陽性となった者のうち,患者である確率

\begin{align}
P_B(A)&=0.03166397
\end{align}
よって$3.17\%$.
\end{frame}


\begin{frame}{目次}
    \tableofcontents
\end{frame}

%コピペ用
%\begin{frame}{ブロック環境}
%    \begin{block}{block}
%    block
%    \end{block}
%    \begin{alertblock}{alertblock}
%    alertblock
%    \end{alertblock}
%    \begin{exampleblock}{exampleblock}
%    exampleblock
%    \end{exampleblock}
%    \begin{tcolorbox}[colframe=green,
%    colback=green!10!white,
%    colbacktitle=green!40!white,
%    coltitle=black, fonttitle=\bfseries,
%    title=My box]
%        box contents
%    \end{tcolorbox}
%\end{frame}

\end{document}
